\documentclass{dtuletter}
\usepackage[utf8]{inputenc}
\usepackage[english]{babel}
\usepackage{verbatim,ifthen,xspace}
\usepackage{array}
\usepackage{longtable}

\newcommand{\optempty}{{\rmfamily\texttt{empty}}}
\newcommand*{\for}[1]{{\rmfamily(#1)}}

\begin{document}

\section*{\Large Industrial IoT for Digitization of Electronic Assets -
Final Project}

\vspace{2em}
\textbf{Project Title:} System Identification of the Wastewater Pump Station in Rønne.

\textbf{Deadline:} 24th of January, 2024


\vspace{1em}
This project aims to explore and analyze the wastewater pump station in Rønne through system identification approach.
The primary goal is to understand the operational dynamics of the station and evaluate various strategies to enhance its performance. 
Students will be divided into three groups, with each group focusing on the same main project but focusing on different objective functions for the controller.

\textbf{Common Objectives:} \\
Each group will work on system identification of the Rønne wastewater pump station.
This will involve gathering data, understanding system dynamics, control and use the digital twin to forecast future scenarios.

\textbf{Unique Group Objectives:} 
\begin{itemize}
	\item \textbf{Group 1: Reducing CO2 emissions.} \\
	This project should focus on reducing the CO2 emission of the station, prioritizing the consumption of the station in case of high renewables production. 
	\item \textbf{Group 2: Overflows Reduction and Integration into the Ancillary Service Market.} \\
	Wastewater stations have pay a penalty each time they overflow. In this project, the group will focus on preventing the overflows by leaveraging the inflow forecast combined with the weather forecast.
	\item \textbf{Group 3: Increase the overall Energy Efficency of the station} \\
	The group will focus in the implementation of a real-time controller than minimize the overall energy consumption of the wastewater pump station.

\end{itemize}

\textbf{Evaluation Criteria:}: \\
\begin{itemize}
	\item[-] Code Implementations, Plots, and Graphs
	\item[-] Quality and accuracy of system identification analysis.
	\item[-] Effectiveness of the proposed control strategy based on the chosen objective function. 
	\item[-] Clarity, Creativity and depth of the presentation.
\end{itemize}
\end{document}